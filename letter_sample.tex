\documentclass[11pt,draft]{mtletter}

\name{First Last}
\address{234 Main St}{Anywhere, CA}{12345}
\phone{123-456-7890}
\email{first.last@gmail.com}

%\KOMAoption{draft=true}
\setkomavar{yourref}{}
\setkomavar{yourmail}{}
\setkomavar{myref}{}
\setkomavar{customer}{}
\setkomavar{invoice}{}


\setkomavar{date}{\today}


\begin{document}

\begin{letter}{%
Professor Cristanne Miller, Chair \\
English Department \\
Pomona College \\
140 W.~6th Street \\
Claremont, CA 91711
}
 
\KOMAoptions{fromphone=true,fromfax=false}
\setkomavar{subject}{Cover Letter}

\opening{Dear Professor Miller,}
 
I am writing to apply for the assistant professorship in British literature of the Victorian period announced in the MLA Job Information List. I am currently completing a Ph.D. at Stanford University in nineteenth- and twentieth-century English literature. My dissertation is entitled "Immense Debtorship: Originality, Literary Property, and Deficit Poetics in British Letters, 1840-1940."

The dissertation investigates the twin discourses of aesthetic and economic value in post-Romantic Britain. My inquiry focuses on originality and imitation--terms that have traditionally belonged to a purely literary domain--in order to implicate them in extra-literary discourses. I illustrate how the modern notion of originality not only appears, but evolves in relation to market forces, economic theory, and the legal developments they underwrite. After linking the cardinal Romantic virtue of originality to classical economic labor theories of value, I trace the nineteenth-century transition from classical political economy to the "marginalist" economics that flourished during the 1870's. This shift from the social scene of production to the individual scene of consumption, I claim, sponsors a concurrent aesthetic shift away from the productivist literary value of originality, with its attendant notions of inspiration, spontaneity, individual genius.

The result is a growing willingness at late-century to imitate, borrow, and even plagiarize--a kind of latter-day neoclassicism that culminates in the canny appropriations of the high modernists. This growth-industry in literary debt, furthermore, is licensed by the gradual decriminalizing of financial insolvency and the institutionalizing of consumer credit during the late nineteenth century: deficit economics begets a "deficit poetics." Many Victorian writers paired debtorship with vampirism, on the grounds that both constituted a necessary pact with evil; my chapter on Eliot's poetry and George Viereck's 1907 novel The House of the Vampire explores the theme and practice of deficit poetics in light of this pairing. Profligate literary debtorship also operates, I argue, as a counter-tradition to the copyright laws solidifying during the period. And for two of the principle figures in this counter-tradition--the Irishmen Oscar Wilde and James Joyce--deficit poetics works as a colonial and post-colonial retort to the British tradition of private annexation embodied in copyright. Contrary to many formalist views, then, there is a politics of intertextuality. By concluding with a look at some late-twentieth-century intellectual property questions (copyright extension bills, electronic media laws, the power of patents to protect corporate appropriations from developing nations), I suggest that mythologies of originality still have considerable ethical and political repercussions in the present. Having written three out of five chapters, I expect to submit the finished thesis in June of 1997.

My scholarly interests range widely, from the history of law and economics to turn-of-the-century sexology to Irish literature and ethnography. In a subsequent project, I plan to research Victorian ideas of order--in both artistic genres and ethnographic taxonomies like Mayhew's London Labour and the London Poor--and how they inform a little-theorized aspect of certain modernist texts: the inventory. In addition, I hope to carry my interests in both Victorian and modernist literature into the classroom. The conventional divide between the two periods too often results in oversimplifications--imputing, say, a gritty materialism to Victorian fiction and a serene formalism to modernist writing. I believe I can help students see not only the ruptures, but the continuities between nineteenth- and twentieth-century literatures and cultures.

My teaching philosophy reflects my interests in collaborative authorship. Instead of the "full frontal teaching" method of large lectures of autocratic seminars, I prefer student-centered teaching that encourages learning by both students and teachers. I favor classroom dynamics that permit dialogue and foster a degree of student input as to curricula and grading criteria. And I like students to think about the class as a community. This means that in both composition and literature classes, I have students spend a fair amount of time in smaller groups in which they not only talk and think together, but write together. In keeping with this emphasis on process, I have used the portfolio grading method in my writing courses, and have been pleased by enthusiastic student reactions. I would be happy to send class syllabi and student evaluations on request.

I have enclosed a copy of my curriculum vitae and a writing sample excerpted from my chapter on the politics of literary property in Wilde. My dossier will arrive under separate cover from the Stanford Career Planning and Placement Center. I will gladly provide any other supporting materials upon request. I plan to attend the MLA convention in Washington D.C. this December, and would be pleased to meet with you there or elsewhere at your convenience. I look forward to hearing from you.

 
\closing{Sincerely,}

\encl{curriculum vitae}
 
\end{letter}

\end{document}

